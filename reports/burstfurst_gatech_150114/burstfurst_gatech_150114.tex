\documentclass{beamer}
\setbeamertemplate{navigation symbols}{}

\usepackage{beamerthemeshadow}
\setbeamertemplate{caption}[numbered]

\hypersetup{colorlinks}

\def\gw#1{gravitational wave#1 (GW#1)\gdef\gw{GW}}
\def\ns#1{neutron star#1 (NS#1)\gdef\ns{NS}}

\newcommand{\red}[1]{{\color{red}{#1}}}

\begin{document}
\title{BHextractor / PCA Update \& Notes}
%\subtitle{Burst Call Oct 8$^{\text{th}}$ 2014}  
\author{James A. Clark}
%\institute{Georgia Institute Of Technology}
\date{} 

\begin{frame}[plain]
\titlepage
\end{frame}

%\begin{frame}\frametitle{Table of contents}\tableofcontents
%\end{frame} 

\section{Projects}

\begin{frame}

    \frametitle{Things we have going on / I know about}
    \begin{itemize}
        \item Continuing BHextractor / Sant-Cugant work
            \begin{itemize} 
                \item `September paper' (Kimbrell)
            \end{itemize}
        \item NR catalogue studies
            \begin{itemize} 
                \item PCAT-style clustering (Magana) 
                \item Waveform / parameter Interpolation (London, Clark)
            \end{itemize}
        \item Analytic Catalogue Studies
            \begin{itemize}
                \item PCA Characterization \& Exploration (Day)
            \end{itemize}
    \end{itemize}

\end{frame}

\section{BHextractor}

\begin{frame}
    \frametitle{BHextractor: previous work}
    \begin{itemize}
        \item Small (~20 waveforms) catalogues (Q, HR, RO3)
        \item Used all available harmonics, constructed optimally oriented
            (face-on) waveforms
        \item PCs computed from truncated, aligned \& SNR-scaled waveforms
        \item Evidence computed in old SMEE matlab script for each catalogue
        \item No sky-location search, no time-search, no mass-search
        \item Single IFO
        \item Reasonable success in distinguishing the more `distinct' waveforms
        \item No attempt at parameter estimation; goal was \emph{only}
            separation of catalogues
    \end{itemize}
\end{frame}

\begin{frame}
    \frametitle{BHextractor: Recent / On-going}
    \begin{itemize}
        \item N. Mangini (now left): started looking at PCs with complex ($h_+ -
            i h_{\times}$) waveforms and handling inclination
        \item D. Leininger (Glasgow summer): investigated sky-localisation and
            multi-IFO application
        \item S. Kimbrell: tentative plan is to repeat \& extend Sant-Cugant
            study using {\tt LALInference} implementation of SMEE.
    \end{itemize}

    Note: we also need
    \begin{enumerate}
        \item careful treatment of orientation \& harmonics
        \item a consistent and meaningful way to define catalogues (do we even
            \emph{want} catalogues?)
        \item Some statement on physical parameters
    \end{enumerate}

\end{frame}

\section{Waveform Clustering}

\begin{frame}
    \frametitle{Waveform Clustering}
    \begin{itemize}
        \item `PCAT': algorithm for glitch classification in detchar:
            \begin{enumerate}
                \item Construct a \emph{single} catalogue of all waveforms
                \item Perform PCA
                \item Use GMM to identify clusters of principle component
                    `scores'
                \item Clusters in PC-score space represent morphologically
                    similar waveforms
                \item So far: prototype script to generate a catalogue
                    comprised of sine-Gaussians \& chirps (randomised params),
                    perform PCA / clustering and identify the different families
                \item \emph{Potential} use in constructing morphologically 
            \end{enumerate}
    \end{itemize}
\end{frame}



\end{document}
